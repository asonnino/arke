\section{Alternative Instantiations} \label{sec:instantiations}
\alberto{Describe how to instantiate \sysname for a centralized system (with crash-fault tolerance), Celo, and Sui. Then refer to the appropriate appendix for details.}

\subsection{Celo}

Celo's mission is to build an inclusive financial system with a seamless user
experience on mobile phones. To maximize user friendliness, one of Celo's main
goals thereby is to enable users to discover each other's cryptocurrency accounts
only via their phone numbers. To realize such a user discovery feature
traditional services often force their users to upload all of their contacts to
a centralized server which is undesirable for a decentralized
cryptocurrency. Another approach could be to utilize Celo's blockchain to do the
matching between phone numbers and cryptocurrency accounts. However, if the
binding between phone numbers and cryptocurrency accounts is done in an obvious
manner, then this can easily lead to privacy and security issues: Since phone
numbers are not secret and Celo's blockchain is publicly accessible, anybody
could check whether a given user has an account on Celo, uncover with whom they
transacted, and perhaps even use the gathered information to mount
spear-phishing attacks.  One approach to mitigate this problem is to obfuscate
this binding and rate-limit the number of discovery requests users can make.
Unfortunately, this impacts the user experience as honest users might not be
able to discover all of their contacts. Rate-limiting also does not prevent
motivated adversaries from mounting Sybil attacks by creating fake accounts and
thereby getting additional discovery requests.

\alberto{Describe how to instatiate \sysname for Celo.}

\subsection{Sui}
\alberto{Explain how to instantiate \sysname for Sui (e.g., how to take advantage of owned objects and why Sui is in a special position to benefit from \sysname).}