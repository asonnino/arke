\section{Introduction} \label{sec:introduction}

% Motivation / problem context
Contact discovery enables users of social applications (like messengers,
payment systems, or media-sharing platforms) to find and interact with their
registered contacts~\cite{marlinspike14contactdiscovery}. 
This process allows to bootstrap social applications on top of an existing
social graph, providing immediate value to the application. 
This is particularly effective when the social graph uses familiar and widely
shared {\em identifiers} such as phone numbers, email addresses or usernames
from popular platforms.

% Why current solutions fall short
Currently deployed solutions fall short of various important expectations,
however: they do not protect users' privacy and expose the underlying
social relations, either by design~\cite{telegrampolicy, whatsapppolicy} or when
subjected to enumeration (or crawling)
attacks~\cite{hagen20discoveryabuse,mozilla21scraping}, they rely on a
centralized party or trusted hardware for privacy
protection~\cite{marlinspike2017technology}, or
they are not scalable enough to be used in globally deployed systems with
billions of users (such as WhatsApp)~\cite{kales2019mobile}.

%Unfortunately, currently-deployed solutions expose the underlying social
%relations either by design~\cite{telegrampolicy, whatsapppolicy} or when
%subjected to user enumeration (or {\em crawling})
%attacks~\cite{hagen20discoveryabuse}, which have been observed in the wild and
%in some cases leaked hundreds of millions of individual user
%records~\cite{mozilla21scraping}.
%The harms incurred by such privacy breaches are well-documented: 
%\begin{quote}
%	"The main privacy issue here is that sensitive contact relationships can become known and could be used to scam, discriminate, or blackmail users, harm their reputation, or make them the target of an investigation. The server could also be compromised, resulting in the exposure of such sensitive information even if the provider is honest". \cite{hagen20discoveryabuse}
%\end{quote}
%
%Such attacks have been observed in the wild, leaking more than 500 million individual records \cite{mozilla21scraping}.
%On the other hand solutions that attempt to protect users' privacy either rely
%on trusted hardware~\cite{TODO} or are not scalable enough to be used in
%globally deployed systems with billions of users like
%WhatsApp~\cite{kales2019mobile}.

% What an ideal system should provide 
This work presents \sysname\footnote{In Greek mythology \sys is the messenger
of the Titans.}, a novel approach to contact discovery that confronts the
shortcomings of existing systems. 
\sysname seeks to protect the privacy of its users, by ensuring unlinkability
of user interactions and by preventing enumeration attacks, while having no
single points of failure or trust and by being scalable with respect to
throughput, latency, bandwidth, and storage requirements. 
%and providing censorship resistance to its users.

% Overview on how we achieve our goals
In brief, \sysname achieves its privacy goals by building the contact discovery
mechanism on top of a novel {\em unlinkable handshake} which enables two users
to establish a shared key pair locally without leaking any connection details to
third parties; 
it avoids single points of failure or trust by integrating this contact
discovery mechanism into a distributed architecture that combines the handshake
protocol with threshold cryptography~\cite{desantis1994how};
and it achieves its scalability goals by carefully choosing and designing its
distributed building blocks to avoid heavy machinery like consensus and instead
only relies on consistent broadcast~\cite{cachin2011introduction} to realize
the communication layer for contact discovery.

% More details on the construction
In \sys, pairs of users, who wish to communicate with each other privately,
create end-to-end encrypted (E2EE) communication channels through which they
can exchange arbitrary payloads for the purpose of a social application, e.g.,
a public key for Signal, or a wallet address for a decentralized payment system.
They do so by first establishing shared secret keys through our new
identity-based non-interactive key exchange (ID-NIKE) protocol and then use
those to locally derive symmetric keys for their \sys E2EE channels. 
The communication is then facilitated by a public, untrusted message board (the
{\em store}). 
While we apply this process for contact discovery, the techniques used in \sys
generalize to wider privacy-preserving applications as the system implements an
unlinkable handshake. 

% Describe properties: private, bidirectional, shared infra
Our novel design provides important privacy and practical properties. 
Firstly, the parties in \sys learn nothing about the users, their payloads
nor who they are communicating with. 
Secondly, \sys enforces bi-directional relations: Alice can only find Bob if
Bob is also looking for Alice. 
This mechanism differs from typical contact discovery schemes and effectively
prevents crawling attacks. 
Finally, \sys allows multiple applications to share the same contact discovery
infrastructure while operating independently and under their own security
assumptions.

% Main insights and building blocks
We instantiate the ID-NIKE using a variant of the scheme defined by Boneh and
Waters~\cite{boneh2013constrained}. Using distributed key
generation~\cite{gennaro2007secure,abraham2021reaching,abraham2022bingo} and
blind threshold BLS signatures~\cite{boldyreva2003threshold,bacho2022on}, we
modify the Boneh-Waters protocol to distribute the master secret key and allow
for oblivious and verifiable key issuance. 
The second insight of \sysname is to design the store in such a way that each
entry can only be written by a single user. 
This allows \sysname to forgo an expensive consensus protocol to maintain
consistency of the store and instead rely on a simpler primitive based on
Byzantine Consistent Broadcast (BCB)~\cite{cachin2011introduction}. 
Finally, \sysname leverages the observation that modern hardware provides every
device with fairly accurate clocks to roughly divide time in a sequence of
epochs. As a result, the \sysname store can be cleared after a fixed number of
epochs without requiring a consensus protocol.

% Performance
We evaluated a prototype of \sysname written in Rust~\cite{matsakis2014rust} on
Amazon EC2 in a geo-distributed wide-area network deployment. ...
\alberto{Add performance info once we have them}

% Contributions
\paragraph{Contributions}
This paper makes the following contributions:
\begin{itemize}
  \item It presents \sysname, a scalable, distributed, and privacy-preserving
    contact discovery system.
  \item It introduces a threshold and oblivious variant of the Boneh-Waters
    identity-based non-interactive key exchange and an unlinkable handshake
    which in combination yield \sysname's core contact discovery protocol.
  \item It shows how \sysname maintains consistency of a distributed key-value
    store without requiring consensus but instead using simpler and more
    efficient broadcast-based primitives.
  \item It provides a full implementation of \sysname and a performance
    evaluation on a real geo-distributed environment and under varying system
    loads.
  \item It shows how two popular blockchains, Celo~\cite{celo} and
    Sui~\cite{sui}, can leverage \sysname to build a privacy-preserving contact
    discovery service for their wallet services, and how messaging services
    such as Signal~\cite{signal}, Telegram~\cite{telegram}, and
    WhatsApp~\cite{whatsapp} can run \sysname to allow users to privately
    discover each other's public keys.
\end{itemize}
The rest of the paper is structured as follows: \Cref{sec:overview} presents an architectural overview of \sysname, \Cref{sec:background} introduces preliminaries and background, \Cref{sec:contact_discovery_protocol} presents our new threshold oblivious ID-NIKE and our unlinkable handshake, \Cref{sec:store} introduces the distributed, consistent, consensus-less \sysname key-value store, \Cref{sec:security} provides a security analysis of \sysname, \Cref{sec:implementation_and_eval} presents our implementation and evaluation of \sysname, \Cref{sec:relatedwork} discusses related work, and \Cref{sec:conclusion} concludes the paper. Additionally, \Cref{sec:instantiations} presents the \sysname instantiations for Celo and Sui.
